\documentclass[10pt]{beamer}
\usetheme[
%%% options passed to the outer theme
%    hidetitle,           % hide the (short) title in the sidebar
%    hideauthor,          % hide the (short) author in the sidebar
%    hideinstitute,       % hide the (short) institute in the bottom of the sidebar
%    shownavsym,          % show the navigation symbols
%    width=2cm,           % width of the sidebar (default is 2 cm)
%    hideothersubsections,% hide all subsections but the subsections in the current section
%    hideallsubsections,  % hide all subsections
%    left                % right of left position of sidebar (default is right)
  ]{Aalborg}
  
% If you want to change the colors of the various elements in the theme, edit and uncomment the following lines
% Change the bar and sidebar colors:
%\setbeamercolor{Aalborg}{fg=red!20,bg=red}
%\setbeamercolor{sidebar}{bg=red!20}
% Change the color of the structural elements:
%\setbeamercolor{structure}{fg=red}
% Change the frame title text color:
%\setbeamercolor{frametitle}{fg=blue}
% Change the normal text color background:
%\setbeamercolor{normal text}{bg=gray!10}
% ... and you can of course change a lot more - see the beamer user manual.

\usepackage[utf8]{inputenc}
\usepackage[english]{babel}
\usepackage[T1]{fontenc}
% Or whatever. Note that the encoding and the font should match. If T1
% does not look nice, try deleting the line with the fontenc.
\usepackage{helvet}


% colored hyperlinks
\newcommand{\chref}[2]{%
  \href{#1}{{\usebeamercolor[bg]{Aalborg}#2}}%
}

\title[Semantic Web Mining]% optional, use only with long paper titles
{Semantic Web Mining}

\subtitle{Semantic Web Course}  % could also be a conference name

\date{\today}

\author[Ali Mohebbi, Mehdi Keshani] 
 % optional, use only with lots of authors
{
  Ali Mohebbi\\
  \href{mailto:a.mohebbi@ce.sharif.edu}{{\tt a.mohebbi@ce.sharif.edu}}\\
   Mehdi Keshani\\
  \href{mailto:keshani@ce.sharif.edu}{{\tt keshani@ce.sharif.edu}}
}

% - Give the names in the same order as they appear in the paper.
% - Use the \inst{?} command only if the authors have different
%   affiliation. See the beamer manual for an example

\institute[
%  {\includegraphics[scale=0.2]{aau_segl}}\\ %insert a company, department or university logo
  Dept.\ of Computer Engineering\\
  Sharif University\\
  Iran
] % optional - is placed in the bottom of the sidebar on every slide
{% is placed on the bottom of the title page
  Dept.\ of Computer Engineering\\
  Sharif University\\
  Iran
  
  %there must be an empty line above this line - otherwise some unwanted space is added between the university and the country (I do not know why;( )
}

% specify the logo in the top right/left of the slide
\pgfdeclareimage[height=1cm]{mainlogo}{AAUgraphics/aau_logo_new} % placed in the upper left/right corner
\logo{\pgfuseimage{mainlogo}}

% specify a logo on the titlepage (you can specify additional logos an include them in 
% institute command below
\pgfdeclareimage[height=1.5cm]{titlepagelogo}{AAUgraphics/aau_logo_new} % placed on the title page
%\pgfdeclareimage[height=1.5cm]{titlepagelogo2}{AAUgraphics/aau_logo_new} % placed on the title page
\titlegraphic{% is placed on the bottom of the title page
  \pgfuseimage{titlepagelogo}
%  \hspace{1cm}\pgfuseimage{titlepagelogo2}
}
\usepackage{ragged2e}
\usepackage{etoolbox}
\usepackage{lipsum}
\apptocmd{\frame}{}{\justifying}{}
\begin{document}
% the titlepage
{\aauwavesbg
\begin{frame}[plain,noframenumbering] % the plain option removes the sidebar and header from the title page
  \titlepage
\end{frame}}
%%%%%%%%%%%%%%%%

% TOC
\begin{frame}{Agenda}{}
\tableofcontents
\end{frame}
%%%%%%%%%%%%%%%%


\section{Data Mining}
\begin{frame}{Data Mining}
\begin{itemize}
\item Data mining is the study of collecting, cleaning, processing, analyzing, and gaining useful
insight from data.
\item Data mining is a broad term that is used to describe different aspects of data processing
(depending on the problem domain, applications, formulation, and data representations).
\item Data may be arbitrary, unstructured, and even in a format that is not immediately suitable
for automated processing.
\end{itemize}
\begin{figure}[H]
	\centering
	\includegraphics[width=0.4\textwidth]{images/DataMining.PNG}
	\label{fig:DataMining}
\end{figure}
\end{frame}

\subsection{Classification}
\begin{frame}{Data Mining}{Classification}
\begin{itemize}
\item Classification is a data mining function that assigns items in a collection to target categories or classes. The goal of classification is to accurately predict the target class for each case in the data. For example, a classification model could be used to identify loan applicants as low, medium, or high credit risks.

\item Classifier include a supervised model that can learn from a training data and then classify the test data.

\item A supervised learning algorithm analyzes the training data and produces an inferred function, which can be used for mapping new examples.

\end{itemize}
\begin{figure}[H]
	\centering
	\includegraphics[width=0.4\textwidth]{images/Classification.PNG}
	\label{fig:Classification}
\end{figure}

\end{frame}
\subsection{Clustring}
\begin{frame}{Data Mining}{Clustering}

\begin{itemize}
\item Clustering is the process of grouping a set of data objects into multiple groups or clusters
so that objects within a cluster have high similarity, but are very dissimilar to objects in
other clusters.
\item Dissimilarities and similarities are assessed based on the attribute values describing the
objects and often involve distance measures.
\item Clustering as a data mining tool has its roots in many application areas such as biology,
security, business intelligence, and Web search.
\end{itemize}

\end{frame}
\subsection{Association Rule Mining}
\begin{frame}{Data Mining}{Association Rule Mining}
\begin{itemize}
\item The classical problem of associative pattern mining is defined in the context of
supermarket (items bought by customers as transactions).
\item The goal is to determine association between groups of items bought by customers.
\item The most popular model for associative pattern mining uses the frequencies of sets of
items as the quantification of the level of association.
\item The discovered set of items are referred to as large itemsets, frequent itemsets, or
frequent patterns.
\begin{figure}[H]
	\centering
	\includegraphics[width=0.4\textwidth]{images/Association.PNG}
	\label{fig:Classification}
\end{figure}
\end{itemize}
\end{frame}
\begin{frame}
\begin{itemize}
\item Frequent itemsets can be used to generate association rules of the form
$X => Y$\\
X and Y are set of items.
\item For example, if the supermarket owner discovers the following rule
{Eggs, Milk} $=>$ {Yogurt}
\item As a conclusion, she/he can promote Yogurt to customers who often buy Eggs and Milk.
\item The frequency-based model for associative pattern mining is very popular due to its
simplicity.

\end{itemize}
\end{frame}
\section{Semantic Web Mining}
\subsection{Classification}
\begin{frame}{Semantic Web Mining}{Classification}

\end{frame}
\subsection{Clustring}
\begin{frame}{Semantic Web Mining}{Clustering}

\end{frame}
\subsection{Association Rule Mining}
\begin{frame}{Semantic Web Mining}{Association Rule Mining}
\end{frame}

\section{Association Rule Mining}
\subsection{Challenges}
\begin{frame}{Association Rule Mining}{Challenges}


\end{frame}
\subsection{Related Works}
\begin{frame}{Association Rule Mining}{Related Works}
\begin{itemize}
\item salam
\item hi
\end{itemize}

\begin{figure}[H]
	\centering
	\includegraphics[width=1.0\textwidth]{images/prediction-process.PNG}
	\caption{process \cite{nam2014survey}}
	\label{fig:prediction-process}
\end{figure}

\end{frame}

\section{References}
 \begin{frame}[allowframebreaks] {References}
 \bibliographystyle{plain}
\bibliography{References}
\end{frame}
%%%%%%%%%%%%%%%%

{\aauwavesbg%
\begin{frame}[plain,noframenumbering]%
  \finalpage{Thank you.}
\end{frame}}
%%%%%%%%%%%%%%%%

\end{document}
